\documentclass[itdr]{subfiles}


\begin{document}
\chapter{Numbered Chapter}
\label{ch:numbered}

\section{def.tex}

The file \lstinline!def.tex! defines some essential variables and should be included in the document's preamble.

\begin{lstlisting}
	\def \game {Into the Dungeon: Revived}	% GAME variable
\def \title {ItDR Core Guide}	% TITLE variable
\def \subtitle {A \LaTeX~style manual for ``\game''}	% SUBTITLE variable
\def \author {Vladimir Arabadzhi}	% AUTHOR variable
\def \keywords {\title;Into the Dungeon;RPG;LaTeX}	% KEYWORDS variable
\def \version {2024/05/20}	% VERSION variable
\def \license {\href{https://creativecommons.org/licenses/by-sa/4.0/}{Creative Commons Attribution-ShareAlike 4.0 International License (CC BY-SA 4.0)}}	% LICENSE variable

% SPECIAL TO ITDR-CORE-GUIDE

\newindex{\spells}{Spell List}
\def\bestiaryIndex{bestiary}
\newindex{bestiary}{Bestiary}

\usepackage{listings}
\lstset
{
	language=[LaTeX]TeX,
	breaklines=true,
	tabsize=2,
	basicstyle=\tt\normalsize,
	keywordstyle=\color{blue},
	identifierstyle=\color{magenta},
	commentstyle=\color{black!65},
	frame = single,
	backgroundcolor=\color{lightgray!50},
}

\newcommand{\iconls}[2]{\lower\IconLower\hbox{\csname #1\endcsname}\hspace*{-0.5em}&\textbackslash#1&#2}

\newcommand{\iconsc}[1]{\lower\IconLower\hbox{\csname #1\endcsname}\hspace*{-0.7em}&{\small\textbackslash#1}}

 % variables
\end{lstlisting}

At the very least, it should have \lstinline!\title!, \lstinline!\author!, and \lstinline!\keywords! definitions, as they are used for the PDF metadata.

\vspace{\baselineskip}
\noindent\begin{minipage}{\textwidth}
	\begin{lstlisting}
		\def \game {Into the Dungeon: Revived}	% GAME variable
		\def \title {ItDR Core Guide}	% TITLE variable
		\def \subtitle {A \LaTeX~style manual for ``\game''}	% SUBTITLE variable
		\def \author {Vladimir Arabadzhi}	% AUTHOR variable
		\def \keywords {\title;Into the Dungeon;RPG;LaTeX}	% KEYWORDS variable
		\def \version {2023/12/01}	% VERSION variable
		\def \license {\href{https://creativecommons.org/licenses/by-sa/4.0/}{Creative Commons Attribution-ShareAlike 4.0 International License (CC BY-SA 4.0)}}	% LICENSE variable
	\end{lstlisting}
\end{minipage}

\vfill

\noindent\begin{minipage}{\textwidth}
\section{Tables}

\begin{multicols}{2}
\header{Header}
\begin{dtable}[L|C|R]
	\textbf{Left-aligned} & \textbf{Centered} & \textbf{Right-aligned}\\
	\rowcolor{dColor1} & \cellcolor{dColor2}text & \cellcolor{dColor2}text\\
	\multirow{-2}{*}{multirow} & text & text \\
	\hline
	text & \multicolumn{2}{c}{multicolumn} \\
	\pcell{multi-line\\paragraph cell} & text & text \\
\end{dtable}

\break

\paragraph{Column alignment}

\begin{itemize}
	\item c --- centred
	\item C --- centred, equal width
	\item l --- left-aligned
	\item L --- left-aligned, equal width
	\item r --- right-aligned
	\item R --- right-aligned, equal width
\end{itemize}

\end{multicols}

\vspace{\baselineskip}
\begin{lstlisting}
\header{Header}
\begin{dtable}[L|C|R]
	\textbf{Left-aligned} & \textbf{Centered} & \textbf{Right-aligned}\\
	\rowcolor{dColor1} & \cellcolor{dColor2}text & \cellcolor{dColor2}text\\
	\multirow{-2}{*}{multirow} & text & text \\
	\hline
	text & \multicolumn{2}{c}{multicolumn} \\
	\pcell{multi-line\\paragraph cell} & text & text \\
\end{dtable}
\end{lstlisting}

\end{minipage}

\vfill
\clearpage

\section{Lists}

\subsection{Itemize}
\begin{itemize}[leftmargin=*]
	\item depth 1, item 1
	\item depth 1, item 2
	\begin{itemize}
		\item depth 2, item 1
		\item depth 2, item 2
		\begin{itemize}
			\item depth 3, item 1
			\item depth 3, item 2
			\begin{itemize}
				\item depth 4, item 1
				\item depth 4, item 2
			\end{itemize}
			\item depth 3, item 3
			\item depth 3, item 4
		\end{itemize}
		\item depth 2, item 3
		\item depth 2, item 4
	\end{itemize}
	\item depth 1, item 3
	\item depth 1, item 4
\end{itemize}

\vfill

\begin{lstlisting}
	\begin{itemize}[leftmargin=*]% no left margin
		\item depth 1, item 1
		\item depth 1, item 2
		\begin{itemize}
			\item depth 2, item 1
			\item depth 2, item 2
			\begin{itemize}
				\item depth 3, item 1
				\item depth 3, item 2
				\begin{itemize}
					\item depth 4, item 1
					\item depth 4, item 2
				\end{itemize}
				\item depth 3, item 3
				\item depth 3, item 4
			\end{itemize}
			\item depth 2, item 3
			\item depth 2, item 4
		\end{itemize}
		\item depth 1, item 3
		\item depth 1, item 4
	\end{itemize}
\end{lstlisting}

\break

\subsection{Enumerate}
\begin{enumerate}
	\item depth 1, item 1
	\itemrange2 depth 1, items 2-3
	\begin{enumerate}
		\item depth 2, item (1)
		\item depth 2, item (2)
		\begin{enumerate}
			\item depth 3, item I
			\itemrange3 items II--IV
			\begin{enumerate}
				\itemrange4 depth 4, items (i--iv)
				\item depth 4, item (v)
			\end{enumerate}
			\item depth 3, item V
			\item depth 3, item VI
		\end{enumerate}
		\item depth 2, item (3)
		\itemrange4 depth 2, items (4--7)
	\end{enumerate}
	\item depth 1, item 4
	\item depth 1, item 5
\end{enumerate}

\vfill

\begin{lstlisting}
	\begin{enumerate}
		\item depth 1, item 1
		\itemrange2 depth 1, items 2-3
		\begin{enumerate}
			\item depth 2, item (1)
			\item depth 2, item (2)
			\begin{enumerate}
				\item depth 3, item I
				\itemrange3 items II--IV
				\begin{enumerate}
					\itemrange4 depth 4, items (i--iv)
					\item depth 4, item (v)
				\end{enumerate}
				\item depth 3, item V
				\item depth 3, item VI
			\end{enumerate}
			\item depth 2, item (3)
			\itemrange4 depth 2, items (4--7)
		\end{enumerate}
		\item depth 1, item 4
		\item depth 1, item 5
	\end{enumerate}
\end{lstlisting}

\cleartoleftpage

\section{Templates}

\subsection{Features}

\noindent\fbox{\lstinline!\\feat\{Feature\}!} Feature

\noindent\fbox{\lstinline!\\feata\{Ancestry\}!} Ancestry

\noindent\fbox{\lstinline!\\featb\{Background\}!} Background

\noindent\fbox{\lstinline!\\feate\{Text\}!} \feate{Text}

\vspace{-1ex}

\noindent\fbox{\lstinline!\\featmt!} \featmt

\vspace{-1ex}

\noindent\fbox{\lstinline!\\feathp!} \feathp

\vspace{-1ex}

\noindent\fbox{\lstinline!\\featadv Text!} \textbf{Advancement:} Text

\feat[\jobname]{Warrior}\featmt\feathp
Gain bonus d4 weapon Damage die.
\featadv{The bonus die is Raised by one.}

\begin{lstlisting}
% indexed
\feat[/jobname]{Warrior}
	\featmt\feathp
Gain bonus d4 weapon Damage die.
\featadv{The bonus die is Raised by one.}

% non-indexed
\feat*{Warrior}
\end{lstlisting}

\feata[\jobname]{Ancestry}
Ancestry description.

\begin{lstlisting}
% indexed
\feata[/jobname]{Ancestry}

% non-indexed
\feata*{Ancestry}
\end{lstlisting}

\featb[\jobname]{Background}
Background description.

\begin{lstlisting}
% indexed
\featb[/jobname]{Background}

% non-indexed
\featb*{Background}
\end{lstlisting}

\break

\subsection{Equipment}

\weapon[\jobname]{Weapon}{10s}{d6/d8}
\begin{lstlisting}
% indexed
\weapon[\jobname]{Weapon}
	{10s}{d6/d8}

% non-indexed
\weapon*{Weapon}{10s}{d6/d8}
\end{lstlisting}

\armour[\jobname]{Armour}{10s}{1}
\begin{lstlisting}
% indexed
\armour[\jobname]{Armour}{10s}{1}

% non-indexed
\armour*{Armour}{10s}{1}
\end{lstlisting}

\equip[\jobname]{Equipment}{10s} Description.
\begin{lstlisting}
% indexed
\equip[\jobname]{Equipment}{10s} Description.

% non-indexed
\equip*{Equipment}{10s} Description.
\end{lstlisting}

\vfill

\subsection{Domain}

\paragraph{Red Hill --- Home of the Man-Beasts}
\domain{Black Yur Og, Veteran Shaman}{5}{Stone walls (8hp, Armour 8), 4 Rock Throwers.\\30 Tribal Champions (2-handed axe), 300 Wild Men (axe, shield), 300 Wild Men (bow).}

\skipline[0.5\baselineskip]

\begin{lstlisting}
\paragraph{Red Hill --- Home of the Man-Beasts}
\domain{Black Yur Og, Veteran Shaman}{5}{Stone walls (8hp, Armour 8), 4 Rock Throwers.\\30 Tribal Champions (2-handed axe), 300 Wild Men (axe, shield), 300 Wild Men (bow).}
\end{lstlisting}

\sizpop{4}

\sizpop{21}

\sizpop[insignificant]{0}

\begin{lstlisting}
\sizpop{4}
\sizpop{21}
\sizpop[insignificant]{0}
\end{lstlisting}

%\vfill
\break

\subsection{Spells}

Default spell index is defined as\\\fbox{\lstinline!\\def\\spells\{spells\}!}.

\noindent Define \fbox{\lstinline!\\newindex\{\\spells\}\{Spell List\}!}\\before the start of your document.

\section{\nth{1} Circle}
\index{Spells!\nth{1} Circle}
\def \spellcircle {1}

\spell[\spells]{Spell} Spell description.
\begin{lstlisting}
\section{\nth{1} Circle}
\index{Spells!\nth{1} Circle}
\def \spellcircle {1}

% indexed
\spell[\spells]{Spell} Spell description.

% non-indexed
\spell*{Spell} Spell description.
\end{lstlisting}

\sspell[\spells]{Stand-alone Spell}{1} Spell description.
\begin{lstlisting}
% indexed
\sspell[\spells]{Stand-alone Spell}{1} Spell description.

% non-indexed
\sspell*{Stand-alone Spell}{1} Spell description.
\end{lstlisting}

\vfill
%\break

\subsection{Monsters}

\statpar[\jobname]{Monster}
\underpar{Underparagraph text}
STR~12, 3hp, Armour~1, d4~Bite, club (d6), abilities.

\subparagraph{Critical Damage in Melee:} Effect.

\begin{lstlisting}
% indexed
\statpar[\jobname]{Monster}

% non-indexed
\statpar*{Monster}

\underpar{Underparagraph text}
STR~12, 3hp, Armour~1, d4~Bite, club (d6), abilities.

\subparagraph{Critical Damage in Melee:} Effect.
\end{lstlisting}
\vspace{-\baselineskip}
\break

\subsection{Miscellaneous}

\paragraph{Subscript and Superscript}

\fbox{\lstinline!\\tsub\{\}!} A \tsub{shorthand} for \fbox{\lstinline!\\textsubscript\{\}!}\\
\fbox{\lstinline!\\tsup\{\}!} A \tsup{shorthand} for \fbox{\lstinline!\\textsuperscript\{\}!}\\

\vspace{-1ex}
\iparagraph{Indexed Paragraph}
\begin{lstlisting}
\iparagraph{Indexed Paragraph}
\end{lstlisting}

\vspace{-1ex}
\paragraph{Saves}

Make a \save{STR}. This affects all \saves{DEX}.

\begin{lstlisting}
Make a \save{STR}. This affects all \saves{DEX}.
\end{lstlisting}

\vspace{-1ex}
\paragraph{Notes}
\vspace{-\baselineskip}
\begin{multicols}{2}
\noindent Text\note[1] \fbox{\lstinline!Text\\note[1]!}\\
Text\note[2] \fbox{\lstinline!Text\\note[2]!}\\
Text\note[3] \fbox{\lstinline!Text\\note[3]!}\\
Text\note[4] \fbox{\lstinline!Text\\note[4]!}\\
Text\note[5] \fbox{\lstinline!Text\\note[5]!}\\
Text\note[6] \fbox{\lstinline!Text\\note[6]!}
\end{multicols}
\footnoterule\noindent
\notetext[1]{Note.} \fbox{\lstinline!\\notetext[1]\{Note\}!}\\
\notetext[2]{Note.} \fbox{\lstinline!\\notetext[2]\{Note.\}!}\\
\notetext[3]{Note.} \fbox{\lstinline!\\notetext[3]\{Note.\}!}\\
\notetext[4]{Note.} \fbox{\lstinline!\\notetext[4]\{Note.\}!}\\
\notetext[5]{Note.} \fbox{\lstinline!\\notetext[5]\{Note.\}!}\\
\notetext[6]{Note.} \fbox{\lstinline!\\notetext[6]\{Note.\}!}\\

\vspace{-2ex}
\paragraph{Footnotes}

\fbox{\lstinline!\\footnotemark[1]!} \fbox{\lstinline!\\footnotetext[1]\{text\}!}

\noindent{}Additionally, see \href{https://www.ctan.org/pkg/sepfootnotes}{sepfootnotes} package.

%\vfill
\paragraph{Symbols}
\fbox{\lstinline!\\dbar!} \dbar~--- prettier double bar.

%\vfill
\paragraph{Lineskip}
\fbox{\lstinline!\\lineskip!} --- skip one line.\\
\fbox{\lstinline!\\lineskip[length]!} --- skip vertical length.

%\vfill
\paragraph{Tight}
\vspace{-0.5\baselineskip}
A shortcut for the \fbox{\lstinline!\\looseness=-1!} command. When placed at the end of a paragraph, it attempts to fit the text in a lesser amount of lines than previously.\tight

\begin{lstlisting}
	Paragraph text above.\tight 
\end{lstlisting}
\vspace{-2\baselineskip}\faHandPointUp

\break

\section{Boxes}

\begin{dbox}
	\paragraph{Default Box}
	Optional rules and notes.\\
	\lipsum[66]
\end{dbox}

\skipline[0.25\baselineskip]

\begin{lstlisting}
\begin{dbox}
	CONTENTS
\end{dbox}
\end{lstlisting}

\vfill
%\break

\begin{abox}
	\paragraph{Aloud Box}
	Room descriptions and other material to read aloud to the players.\\
	\lipsum[66]\tight
\end{abox}

\skipline[0.25\baselineskip]

\begin{lstlisting}
\begin{abox}
	CONTENTS
\end{abox}
\end{lstlisting}

\vfill

\begin{qbox}
	\paragraph{Quote Box}
	Quotes and handouts.\\
	\lipsum[66]\tight
\end{qbox}

\skipline[0.25\baselineskip]

\begin{lstlisting}
\begin{qbox}
	CONTENTS
\end{qbox}
\end{lstlisting}

\vfill
\break

\begin{cbox}
	\paragraph{Centered Box}
	Short quotes and handouts.
\end{cbox}

\begin{lstlisting}
\begin{cbox}
	CONTENTS
\end{cbox}
\end{lstlisting}

\vfill

\begin{bbox}
	\paragraph{Background Box}
	Highlighting important information.\\
	\lipsum[66]
\end{bbox}

\skipline[0.5\baselineskip]

\begin{lstlisting}
\begin{bbox}
	CONTENTS
\end{bbox}
\end{lstlisting}

\vfill

\begin{lbar}
	\paragraph{Left Bar}
	\lipsum[66]\tight
\end{lbar}

\skipline[0.25\baselineskip]

\begin{lstlisting}
	\begin{lbar}
		CONTENTS
	\end{lbar}
\end{lstlisting}

\vfill

\paragraph{Right Box}
\rightbox[\parskip]{\faHandPointRight~\em{SOME TEXT}}
Put a right-aligned line of text raised by \fbox{\lstinline!\\baselineskip + #1!}.
Default of \fbox{\lstinline!#1!} is \fbox{\lstinline!\\parskip!} to put the text on a \fbox{\lstinline!\\paragraph!} line above.

\skipline[0.25\baselineskip]

\begin{lstlisting}
\paragraph{Right Box}
\rightbox[\parskip]{%
	\faHandPointRight~\em{SOME TEXT}}
TEXT
\end{lstlisting}

\vfill

\paragraph{Right Text}

Right-align a part \righttext{of the text.}

\begin{lstlisting}
Right-align a part \righttext{of the text.}
\end{lstlisting}

\vfill
\break

\section{Icons}

\icon{\faGem} Icons from the \href{https://ctan.org/pkg/fontawesome5}{fontawesome5} package.

\begin{lstlisting}
\icon{\faGem} Icons from the \href{https://ctan.org/pkg/fontawesome5}{fontawesome5} package.
\end{lstlisting}

\skipline

\icontitle\faGem\paragraph{Icon and Paragraph}

\begin{lstlisting}
\icontitle\faGem\paragraph{Icon and Paragraph}
\end{lstlisting}

Print an icon at the start of the title line.

\skipline

\header{Examples}
\begin{dtable}[cl|L]

\iconls{faBolt}{trigger}\\

\iconls{faBookmark}{bookmark}\\

\iconls{faCube}{container}\\

\iconls{faDragon}{monster}\\

\iconls{faDungeon}{door}\\

\iconls{faExclamationTriangle}{encounter}\\

\iconls{faEye}{visible}\\

\iconls{faFlag}{flag}\\

\iconls{faGem}{treasure}\\

\iconls{faHiking}{travel}\\

\iconls{faHourglassHalf}{timer}\\

\iconls{faKey}{key}\\

\iconls{faOldKey}{\em (custom v4 icon)}\\

\iconls{faLock}{locked}\\

\iconls{faMagic}{magic}\\

\iconls{faMap}{map}\\

\iconls{faMapSigns}{directions} \\

\iconls{faMoon}{night}\\

\iconls{faPaw}{tracks}\\

\iconls{faRulerCombined}{size, dimensions}\\

\iconls{faScroll}{scroll, info}\\

\iconls{faSearch}{hidden}\\

\iconls{faSkullCrossbones}{trap, danger}\\

\iconls{faStar}{special}\\

\iconls{faSun}{day}\\

\iconls{faTrophy}{reward}\\

\iconls{faUser}{character}\\

\iconls{faUsers}{group}\\

\end{dtable}

\vfill
\break

\header{Shortcuts}

\begin{dtable}[cL|cL]
	\iconItDR & \multicolumn{3}{l}{\textbackslash iconItDR\note} \\
	\iconsc{iconCharacter}	& \iconsc{iconMagic}	\\
	\iconsc{iconContainer}	& \iconsc{iconMonster}	\\
	\iconsc{iconDanger}		& \iconsc{iconNight}	\\
	\iconsc{iconDay}		& \iconsc{iconReward}	\\
	\iconsc{iconDirections}	& \iconsc{iconSize}		\\
	\iconsc{iconDoor}		& \iconsc{iconStar}		\\
	\iconsc{iconEncounter}	& \iconsc{iconTimer}	\\
	\iconsc{iconGroup}		& \iconsc{iconTravel}	\\
	\iconsc{iconHidden}		& \iconsc{iconTreasure}	\\
	\iconsc{iconKey}\note	& \iconsc{iconTrigger}	\\
	\iconsc{iconLock}		& \iconsc{iconVisible}	\\
\end{dtable}

\note {\em Custom icon declared in \lstinline!itdr.sty!}

\vfill

\section{Links}
\label{sec:links}

\begin{lstlisting}
\label{sec:links}
\end{lstlisting}

\subsubsection{Customref}

\customref{sec:links}{A link to Links section}

\begin{lstlisting}
\customref{sec:links}{A link to Links section}
\end{lstlisting}

\subsubsection{Fullref}

\fullref{sec:links}

\begin{lstlisting}
\fullref{sec:links}
\end{lstlisting}

\subsubsection{Safenameref}

\safenameref{sec:links}{Links section} and \safenameref{sec:other}{some other section}.

If the label is undefined, fallback is used.

\begin{lstlisting}
\safenameref{sec:links}{Links section} and \safenameref{sec:other}{some other section}.
\end{lstlisting}

\subsubsection{Safepageref}

\safepageref{Links: p.}{sec:links}
	{.}{Links: fallback.}\\
\safepageref{Other: p.}{sec:other}
	{.}{Other: fallback.}

Output: \lstinline!before ?? after!

Where \lstinline!??! is a page number of the label. If label is undefined, fallback is used.

\begin{lstlisting}
\safepageref{Links: p.}{sec:links}
		{.}{Links: fallback}\\
\safepageref{Other: p.}{sec:other}
		{.}{Other: fallback}
\end{lstlisting}

\cleartoleftpage

\begin{minipage}{\textwidth}
\section{Wrap}

Using the \mbox{\href{https://www.ctan.org/pkg/wrapfig2}{wrapfig2}} package (imported in the \lstinline!itdr.sty!).

\begin{lstlisting}
% FIGURE
\begin{wrapfigure}[indented lines number]{location}[overhang][width]
	figure
\end{wrapfigure}

% TABLE
\begin{wraptable}[indented lines number]{location}[overhang][width]
	table
\end{wraptable}

% TEXT
\begin{wraptext}[indented lines number correction]{location}[overhang]{width}
	(optional style settings)
	\includeframedtext[insertion measure]{text to frame}[settings][radius]
\end{wraptext}

% FLOAT (underlying environment)
\begin{wrapfloat}{object name}[line number]{location}[overhang]{width}
	object
\end{wrapfloat}
\end{lstlisting}

New environment (around \lstinline!wrapfloat!):

\begin{lstlisting}
\begin{wrap}[line number]{location}[overhang]{width}
	object
\end{wrap}
\end{lstlisting}

\paragraph{Location}

\begin{itemize}
	\item l --- left
	\item L --- floating left
	\item r --- right
	\item R --- floating right
	\item i --- inner margin
	\item I --- floating inner margin
	\item o --- outer margin
	\item O --- floating outer margin
\end{itemize}


\end{minipage}

\clearpage

\subsection{Wrap Examples}

\begin{wrap}{l}{0.5\linewidth}
\vspace{-\baselineskip}
\subsubsection{Left Wrap}
\begin{lstlisting}
\begin{wrap}{l}
	{0.5\linewidth}
	% CONTENT
\end{wrap}
\end{lstlisting}
\end{wrap}

\lipsum[1]

\begin{wrap}{r}[0.5\linewidth]{\linewidth}
\vspace{-\baselineskip}
\subsubsection{Central Wrap}
\begin{lstlisting}
% first column
\begin{wrap}{r}
	[0.5\linewidth]{\linewidth}
	\vspace{-\baselineskip}
	% CONTENT
\end{wrap}

% second column
% (empty wrap environment to match
% the one in the first column)
\begin{wrap}[15]{l}{0.46\linewidth}
\end{wrap}
\end{lstlisting}
\end{wrap}

\skipline
\lipsum[2]

\skipline
\lipsum[3]

\begin{wrap}{r}{0.5\linewidth}
\vspace{-\baselineskip}
\subsubsection{Right Wrap}
\begin{lstlisting}
\begin{wrap}{r}
	{0.5\linewidth}
	% CONTENT
\end{wrap}
\end{lstlisting}
\end{wrap}

\skipline
\lipsum[4]

\begin{wrap}[15]{l}{0.46\linewidth}
\end{wrap}

\skipline
\lipsum[5]

\vfill
\cleartoleftpage

\dimagebottom{dimagebottom}{150pt}

\section{Images}

\dimage{dimage}{400pt}

\break

The default \lstinline!\graphicspath! is:\\
\fbox{\lstinline!\{./img/\} \{./img/pic/\} \{./pic/\} \{./itdr/img/\}!}\\

\vspace{-0.5ex}
\begin{lstlisting}
\dimage{dimage}{400pt}
% \dimage[options]{filename}[extension]{height}
\end{lstlisting}
\faHandPointLeft

\vfill

\dimage[trim={0pt 110pt 0pt 0pt},clip]{itdr_logo}[pdf]{100pt}
\faHandPointUp
\begin{lstlisting}
\dimage[trim={0pt 110pt 0pt 0pt},clip]{itdr_logo}[pdf]{100pt}
% [trim=\{left bottom right top\}, clip]
\end{lstlisting}

\vfill

\begin{lstlisting}
\thispagestyle{empty}
\dimagepage{dimagepage}
% \dimagepage[options]{filename}[extension]
\end{lstlisting}
{\em (See the next page)}\hfill\faHandPointRight

\vfill

\begin{lstlisting}
% place BEFORE anything on the page
\dimagebottom{dimagebottom}{150pt}
% \dimagebottom[options]{filename}[extension]{height}
\end{lstlisting}
\faHandPointDown
\vspace{-3.5ex}

\break

\thispagestyle{empty}
\dimagepage{dimagepage}

\end{document}
