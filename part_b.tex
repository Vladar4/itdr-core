\documentclass[itdr]{subfiles}

\begin{document}

\chapterx{Appendix B: Bestiary}
\label{ch:appendix_b}

\noindent\begin{minipage}{\textwidth}
\section{Bestiary package}

\subsection{Database fields}

\fbox{\lstinline!bestiary.csv!} database fields (separated by \fbox{\lstinline!|!} character):

\noindent\begin{lstlisting}
	       |-----------stat-----------|    |--info--(upt to 10 info fields)
         |                          |    |
1   |2   |3  |4  |5  |6 |7     |8   |9   |10     |
    |    |   |   |   |  |      |    |    |       |
name|tags|str|dex|wil|hp|armour|data|desc|info(1)|...
\end{lstlisting}

\paragraph{Fields}
\begin{enumerate}
	\item \fbox{\lstinline!name!} --- monster name
	\item \fbox{\lstinline!tags!} --- comma-separated list of tags (beast, construct, demon, dragon, monster, person, swarm, undead, etc.)
	\itemrange3 \fbox{\lstinline!str!}, \fbox{\lstinline!dex!}, \fbox{\lstinline!wil!} --- ability scores, 10 by default
	\item \fbox{\lstinline!hp!} --- hit points
	\item \fbox{\lstinline!armour!} --- armour score and description, if needed, e.g.: 2 (piecemeal armour + shield)
	\item \fbox{\lstinline!data!} --- the rest of the first line of a stat block: attacks, abilities, etc.
	\item \fbox{\lstinline!desc!} --- literary description paragraph
	\item \fbox{\lstinline!info!} --- ten fields for the abilities and properties descriptions
\end{enumerate}

\paragraph{Example}
\noindent\begin{lstlisting}
name|tags|str|dex|wil|hp|armour|data|desc|info||
Elephant|beast|20||8|12|1|d10~Tusks||\subparagraph{Charge:} a target must pass a \save{DEX} or take Tusks Damage and be knocked prone.|\subparagraph{Trample:} a prone target takes d12~Damage.|
\end{lstlisting}
\end{minipage}

\vfill
\clearpage

\noindent\begin{minipage}{\textwidth}
\subsection{Managing the database}

\begin{itemize}
	\item \fbox{\lstinline!\\bestiaryDefault!} --- default database file name: \fbox{\lstinline!itdr/bestiary!}
	\item \fbox{\lstinline!\\bestiaryExtension!} --- database file extension: \fbox{\lstinline!csv!}
	\item \fbox{\lstinline!\\bestiaryFilename\{filename\}!} --- add the file extension to a file name (done automatically on load)
	\item \fbox{\lstinline!\\def\\bestiaryIndex\{\}!} --- (default) disable the bestiary index
	\item \fbox{\lstinline!\\def\\bestiaryIndex\{\\jobname\}!} --- set the bestiary index to default one.
	\item \fbox{\lstinline!\\def\\bestiaryIndex\{bestiary\}!} --- (used in this document) set the bestiary index to \lstinline!bestiary!.
	\item \fbox{\lstinline!\\usepackage\{itdr/bestiary\}!} --- load with the default database file: \fbox{\lstinline!itdr/bestiary.csv!}
	\item \fbox{\lstinline!\\usepackage[db=filename]\{itdr/bestiary\}!} --- load with a custom database file
	\item \fbox{\lstinline!\\usepackage[db=\{file1, file2, file3, ...\}]\{itdr/bestiary\}!} --- load with a custom list of database files
	\item \fbox{\lstinline!\\bestiaryClear!} --- clear the database
	\item \fbox{\lstinline!\\bestiarySort!} --- sort the database by name field
	\item \fbox{\lstinline!\\bestiaryLoad\{filename\}!} --- clear, load and sort the database by name field
	\item \fbox{\lstinline!\\bestiaryAppend\{filename\}!} --- append to the database and sort it by name field
	\item \fbox{\lstinline!\\bestiaryLoadList\{file1, file2, file3, ...\}!} --- clear the database, load a comma-separated list of files, and then sort it by name field
	\item \fbox{\lstinline!\\DTLdef\{\\name\}\{bestiaryDB\}\{column1 name\}\{column1 value\}\{column2 name\}!} --- store the value of column2 into a new \textbackslash{}def
\end{itemize}

\end{minipage}

\vfill
\clearpage

\section{Print Names}

\begin{lstlisting}
\bestiaryPrintNamesByTag
	{tag}[prefix][suffix]

\bestiaryPrintNamesRange
	{first}{last}[prefix][suffix]

\bestiaryPrintNamesRangeByTag
	{first}{last}{tag}[prefix][suffix]

\bestiaryPrintAllNames
	[prefix][suffix]
\end{lstlisting}

\begin{lstlisting}
\begin{enumerate}
	\bestiaryPrintNamesByTag{beast}
		[\item]
\end{enumerate}
\end{lstlisting}

\begin{enumerate}
	\bestiaryPrintNamesByTag{beast}[\item]
\end{enumerate}

\vfill

\begin{lstlisting}
\begin{enumerate}
	\bestiaryPrintNamesRange
		{Landshark}{Manticore}
			[\item]
\end{enumerate}
\end{lstlisting}

\begin{enumerate}
	\bestiaryPrintNamesRange{Landshark}{Manticore}[\item]
\end{enumerate}

\vfill

\begin{lstlisting}
	\begin{enumerate}
		\bestiaryPrintNamesRangeByTag
			{Landshark}{Manticore}{beast}
				[\item]
	\end{enumerate}
\end{lstlisting}

\begin{enumerate}
	\bestiaryPrintNamesRangeByTag{Landshark}{Manticore}{beast}[\item]
\end{enumerate}

\vfill
\break

\begin{lstlisting}
	\begin{enumerate}
		\bestiaryPrintAllNames[\item]
	\end{enumerate}
\end{lstlisting}

\begin{enumerate}
	\bestiaryPrintAllNames[\item]
\end{enumerate}

\vfill
\break

\section{Print an Entry}

\begin{lstlisting}
\bestiaryPrint[fields]{name}

\bestiaryPrint*[fields]{name}

\bestiaryPrintStat{name}

% the starred version does not write into the index
\end{lstlisting}

\paragraph{Fields:}
\begin{itemize}
	\item \lstinline!name! --- header and index entry
	\item \lstinline!stat! --- main stat block
	\item \lstinline!desc! --- description
	\item \lstinline!info! --- abilities, properties, etc.
\end{itemize}

\skipline
\hrule
\vfill

\begin{lstlisting}
\bestiaryPrint{Gazer}
\end{lstlisting}

\bestiaryPrint{Gazer}

\vfill

\begin{lstlisting}
\bestiaryPrint[name,desc]{Gazer}
\end{lstlisting}

\bestiaryPrint[name,desc]{Gazer}

\vfill

\begin{lstlisting}
\subparagraph{Gazer}
	(\bestiaryPrintStat{Gazer})
\end{lstlisting}

\subparagraph{Gazer} (\bestiaryPrintStat{Gazer})

\vfill
\break

\section{Print Entries}

\begin{lstlisting}
\bestiaryPrintEntriesByTag[fields]
	{tag}[before][after]

\bestiaryPrintEntriesByTag*[fields]
	{tag}[before][after]

\bestiaryPrintEntriesRange[fields]
	{first}{last}[before][after]

\bestiaryPrintEntriesRange*[fields]
	{first}{last}[before][after]

\bestiaryPrintEntriesRangeByTag
	[fields]{first}{last}{tag}
		[before][after]

\bestiaryPrintEntriesRangeByTag*
	[fields]{first}{last}{tag}
		[before][after]

\bestiaryPrintAllEntries
	[before][after][fields]

\bestiaryPrintAllEntries*
	[before][after][fields]

% the starred versions do not write into the index
\end{lstlisting}

\begin{comment}
\subsection{By Tag (monster)}

\begin{lstlisting}
\bestiaryPrintEntriesByTag{monster}
	[][\vfill]
\end{lstlisting}

\bestiaryPrintEntriesByTag{monster}[][\vfill]
\end{comment}

\vfill

\subsection{By Tag (beast)}

\begin{lstlisting}
\bestiaryPrintEntriesByTag
	[name,stat]{beast}[][\vfill]
\end{lstlisting}

\bestiaryPrintEntriesByTag[name,stat]{beast}[][\vfill]

\vfill

\subsection{Range}

\begin{lstlisting}
\bestiaryPrintEntriesRange
	[name,stat]{Landshark}{Manticore}
\end{lstlisting}

\bestiaryPrintEntriesRange[name,stat]{Landshark}{Manticore}

\vfill

\subsection{Range By Tag (beast)}

\begin{lstlisting}
\bestiaryPrintEntriesRangeByTag
	{Landshark}{Manticore}{beast}
\end{lstlisting}

\bestiaryPrintEntriesRangeByTag{Landshark}{Manticore}{beast}

\vfill
\break

\section{Print All Entries}

\begin{lstlisting}
\bestiaryPrintAllEntries[][\vfill]
\end{lstlisting}

\bestiaryPrintAllEntries[][\vfill]

\end{document}
